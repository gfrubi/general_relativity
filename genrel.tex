\documentclass{genrel}

%\includeonly{ch01/ch01temp}

\makeindex
\pdfmapfile{=fullembed.map} % created by the script create_fullembed_file

\begin{document}

\include{cover/cover}

\cleardoublepage

\include{cover/inside-title-page}

\include{cover/copyright}

\pagebreak\vspace{100mm}

\hbox{}\noindent\huge\bfseries\sffamily{}\hspace{-2mm}\ \ Brief Contents\\
\hspace{-20mm}\noindent\mynormaltype\Large\sffamily{}\begin{tabular}{rl}
\ref{ch:intro} & A Geometrical Theory of Spacetime \quad \pageref{ch:intro}\\
\ref{ch:flat} & Geometry of Flat Spacetime \quad \pageref{ch:flat}\\
\ref{ch:differential-geometry} & Differential Geometry \quad \pageref{ch:differential-geometry}\\
\ref{ch:tensors} & Tensors \quad \pageref{ch:tensors}\\
\ref{ch:curvature} & Curvature \quad \pageref{ch:curvature}\\
\ref{ch:vacuum} & Vacuum Solutions \quad \pageref{ch:vacuum}\\
\ref{ch:symmetries} & Symmetries \quad \pageref{ch:symmetries}\\
\ref{ch:sources} & Sources \quad \pageref{ch:sources}\\
\ref{ch:waves} & Gravitational Waves \quad \pageref{ch:waves}\\
\end{tabular}
\mynormaltype

\vspace{100mm}\pagebreak

\cleardoublepage

\noindent\huge\bfseries\sffamily{}\hspace{-2mm}\ \ Contents\\
\mynormaltype

\tableofcontents

\vspace{5mm}

\noindent {\noindent
{\sffamily{} Appendix \ref{einstein-papers}: Excerpts from three papers by Einstein \dotfill \pageref{einstein-papers}}\\
``On the electrodynamics of moving bodies'' \dotfill \pageref{einstein-relativity}\\
``Does the inertia of a body depend upon its energy content?'' \dotfill \pageref{einstein-paper-mc2}\\
``The foundation of the general theory of relativity'' \dotfill \pageref{einstein-foundation}\\
{\sffamily{} Appendix \ref{hwansappendix}: Hints and solutions                      \dotfill \pageref{hwansappendix}}
}

%========================= main matter =========================
\mainmatter
%-- I want the whole book numbered sequentially, arabic:
  \pagenumbering{arabic} 
  \addtocounter{page}{10} 
\parafmt
\myeqnspacing % Do this early and often, since it gets reset by \normalsize
%========================= chapters =========================
	\renewcommand{\chapdir}{ch01}\include{ch01/ch01temp}
	\renewcommand{\chapdir}{ch02}\include{ch02/ch02temp}
	\renewcommand{\chapdir}{ch03}\include{ch03/ch03temp}
	\renewcommand{\chapdir}{ch04}\include{ch04/ch04temp}
	\renewcommand{\chapdir}{ch05}\include{ch05/ch05temp}
	\renewcommand{\chapdir}{ch06}\include{ch06/ch06temp}
	\renewcommand{\chapdir}{ch07}\include{ch07/ch07temp}
	\renewcommand{\chapdir}{ch08}\include{ch08/ch08temp}
	\formatchtoc{\large}{\quad\contentspage}{4mm} % This has to go before the last chapter.
	\renewcommand{\chapdir}{ch09}\include{ch09/ch09temp}

%=================================================================================================================================

\backmatter
\nomarginlayout   
\formatchtoc{\large}{\quad\contentspage}{0mm} % This has to go before first appendix.
\renewcommand{\chaptermark}[1]%
    {\markboth{\textsf{\thechapter\hspace{\myfooterspace}#1}}{}}
\blankchaptermarks
\vfill\pagebreak
% Einstein papers
        \onecolumn\include{ch99/einstein}
\vfill\pagebreak
% hw hints, answers, solutions
        \onecolumn\refstepcounter{appendixctr}\label{hwansappendix}%
\appendix\chapter{Appendix \ref{hwansappendix}: Hints and Solutions}
	
%==================================================================
%==================================================================
%========================= Solutions ==============================
%==================================================================
%==================================================================

\hwanssection{Solutions to Selected Homework Problems}

\beginsolutions{1}

\hwsolnhdr{ordered-geom-finite-models}

Pick two points P1 and P2. By O2, there is another point P3 that is distinct
from P1 and P2. (Recall that the notation [ABC] was defined so that all three
points must be distinct.) Applying O2 again, there must be a further point
P4 out beyond P3, and by O3 this can't be the same as P1. Continuing in this
way, we can produce as many points as there are integers.

\hwsolnhdr{have-spacesuit}

(a) If the violation of (1) is tiny, then of course Kip won't really have any
practical way to violate (2), but the idea here is just to illustrate the
idea, so to make things easy, let's imagine an unrealistically large violation
of (1). Suppose that neutrons have about the same inertial mass as protons, but
zero gravitational mass, in extreme violation of (1). This implies that neutron-rich
elements like uranium would have a much lower gravitational acceleration on earth
than ones like oxygen that are roughly 50-50 mixtures of neutrons and protons.
Let's also simplify by making a second unrealistically extreme assumption: let's
say  that Kip has a keychain in his pocket made of neutronium, a substance composed of
pure neutrons. On earth, the keychain hovers in mid-air. Now he can release
his keychain in the prison cell. If he's on a planet, it will hover.
If he's in an accelerating spaceship, then the keychain will follow Newton's
first law (its tendency to do so being measured by its nonzero inertial mass),
while the deck of the ship accelerates up to hit it.

(b) It violates O1. O1 says that objects prepared in identical inertial states
(as defined by two successive events in their motion) are predicted to have
identical motion in the future. This fails in the case where Kip releases the
neutronium keychain side by side with a penny.

\hwsolnhdr{ep-charge}
(a) In case 1 there is no source of energy, so the particle cannot radiate.
In case 2-4, the particle radiates, because there are sources of energy (loss of
gravitational energy in 2 and 3, the rocket fuel in 4).

(b) In 1, Newton says the object is subject to zero net force, so its motion
is inertial. In 2-4, he says the object is subject to a nonvanishing net force,
so its motion is noninertial. This matches up with the results of the energy analysis.

(c) The equivalence principle, as discussed on page \pageref{sec:chiao-paradox},
is vague, and is particularly difficult to apply successful and unambiguously to
situations involving electrically charged objects, due to the difficulty of
defining locality. Applying the equivalence principle in the most naive way,
we predict that there can be no radiation in cases 2 and 3 (because the object is
following a geodesic, minding its own business).
In case 4, everyone agrees that there will be radiation observable back on earth
(although it's possible that it would not be observable to an observer momentarily
matching velocities with the rocket).
The naive equivalence principle says that 1 and 4 must give the same result, so
we should have radiation in 1 as well. These predictions are wrong in two out of
the four equations, which tells us that we had better either not apply the equivalence
principle to charged objects, or not apply it in such a naive way.

\beginsolutions{2}

\hwsolnhdr{clock-postulate}

(a) Let $t$ be the time taken in the lab frame for the light to go from one mirror to the other,
and $t'$ the corresponding interval in the clock's frame. Then $t'=L$, and $(vt)^2+L^2=t^2$,
where the use of the same $L$ in both equations makes use of our prior knowledge that there
is no transverse length contraction.
Eliminating $L$, we find the expected expression for $\gamma$, which is independent of $L$
(b) If the result of a were independent of $L$, then the relativistic time dilation would depend
on the details of the construction of the clock measuring the time dilation. We would be forced
to abandon the geometrical interpretation of special relativity.
(c) The effect is to replace $vt$ with $vt+at^2/2$ as the quantity inside the parentheses
in the expression $(\ldots)^2+L^2=t^2$. The resulting correction terms are of higher order in
$t$ than the ones appearing in the original expression, and can therefore be made as small
in relative size as desired by shortening the time $t$. But this is exactly what happens when
we make the clock sufficiently small.

\hwsolnhdr{sagnac-area}

(a) Reinterpret figure \figref{thomas-as-area} on p.~\pageref{fig:thomas-as-area} as a picture of a Sagnac
ring interferometer. Let light waves 1 and 2 move around the loop in opposite senses. Wave 1 takes time
$t_{1i}$ to move inward along the crack, and time $t_{1o}$ to come back out. Wave 2 takes times
$t_{2i}$ and $t_{2o}$. But $t_{1i}=t_{2i}$ (since the two world-lines are identical), and similarly
$t_{1o}=t_{2o}$. Therefore creating the crack has no effect on the interference between 1 and 2,
and splitting the big loop into two smaller loops merely splits the total phase shift between them.
(b) For a circular loop of radius $r$, the time of flight of each wave is proportional to $r$, and
in this time, each point on the circumference of the rotating interferometer travels a distance
$v(\text{time})=(\omega r)(\text{time})\propto r^2$. (c) The effect is proportional to area, and
the area is zero. (d) The light clock in c has its two ends synchronized according to the Einstein
prescription, and the success of this synchronization verifies Einstein's assumption of commutativity
in this particular case. If we make a Sagnac interferometer in the shape of a triangle, then the Sagnac
effect measures the failure of Einstein's assumption that all three corners can be synchronized
with one another.

\hwsolnhdr{velocity-addition-matrix-taylor}

Here is the program:
\begin{listing}{1}
L1:matrix([cosh(h1),sinh(h1)],[sinh(h1),cosh(h1)]);
L2:matrix([cosh(h2),sinh(h2)],[sinh(h2),cosh(h2)]);
T:L1.L2;
taylor(taylor(T,h1,0,2),h2,0,2);
\end{listing}
The diagonal components of the result are both $1+\eta_1^2/2+\eta_2^2/2+\eta_1\eta_2+\ldots$
Everything after the 1 is nonclassical. The 
off-diagonal components are $\eta_1+\eta_2+\eta_1\eta_2^2/2+\eta_2\eta_1^2/2+\ldots$,
with the third-order terms being nonclassical.

\beginsolutions{3}

\hwsolnhdr{carousel-paradox}

The process that led from the Euclidean metric of example \ref{eg:metric-in-polar-coords} on page \pageref{eg:metric-in-polar-coords}
to the non-Euclidean one of equation [\ref{eq:rotating-spatial}] on page \pageref{eq:rotating-spatial} was not just a series
of coordinate transformations. At the final step, we got rid of the variable $t$, reducing the number of dimensions by one.
Similarly, we could take a Euclidean three-dimensional space and eliminate all the points except for the ones on the surface
of the unit sphere; the geometry of the embedded sphere is non-Euclidean, because we've redefined geodesics to be lines that
are ``as straight as they can be'' (i.e., have minimum length) while restricted to the sphere. In the example of the carousel, the final step effectively
redefines geodesics so that they have minimal length as determined by a chain of radar measurements.

\hwsolnhdr{carousel-metric}

(a) The $\der\theta'^2$ term of the metric blows up here. A geodesic connecting point A, at $r=1/\omega$, with point B, at $r<1/\omega$,
must have minimum length. This requires that the geodesic be directly radial at A, so that $\der\theta'=0$; for if not, then we could
vary the curve slightly so as to reduce $|\der\theta'|$, and the resulting increase in the $\der r^2$ term would be negligible
compared to the decrease in the $\der\theta'^2$ term. (b) The spatial track of a laser beam is nor a geodesic of this metric.
For example, a laser beam sent outward from the axis would make a track that was straight in the lab frame, but curved in the
rotating frame. Since the spatial metric in the rotating frame is symmetric with respect to clockwise and counterclockwise,
the metric can never result in geodesics with a specific handedness.

\beginsolutions{4}

\hwsolnhdr{lhc-proton-speed}
To avoid loss of precision in numerical operations like subtracting $v$ from $1$,
it's better to derive an ultrarelativistic approximation. The velocity corresponding
to a given $\gamma$ is $v=\sqrt{1-\gamma^{-2}}\approx 1-1/2\gamma^2$, so
$1-v\approx 1/2\gamma^2=(m/E)^2/2$. Reinserting factors of $c$ so as to make the units
come out right in the SI system, this becomes $(mc^2/E)^2/2=9\times 10^{-9}$.

\hwsolnhdr{doppler-three-d}

A spatial plane is determined by the light's direction of propagation and the relative velocity
of the source and observer, so the 3+1 case reduces without loss of generality to 2+1 dimensions.
The frequency four-vector must be lightlike, so its most general possible form is
$(f,f\cos\theta,f\sin\theta)$, where $\theta$ is interpreted as the angle between the direction of
propagation and the relative velocity. Putting this through a Lorentz boost along the $x$ axis,
we find $f'=\gamma f(1+v\cos\theta)$, which agrees with Einstein's equation on page
\pageref{einstein-doppler}, except for the arbitrary convention involved in defining the sign of $v$.

\beginsolutions{6}

\hwsolnhdr{carousel-singularities} 
(a) There are singularities at $r=0$, where $g_{\theta'\theta'}=0$, and $r=1/\omega$,
where $g_{tt}=0$. These are considered singularities because the inverse of the metric
blows up. They're coordinate singularities, because they can be removed by a change of
coordinates back to the original non-rotating frame.\\
(b) This one has singularities in
the same places. The one at $r=0$ is a coordinate singularity, because at small $r$
the $\omega$ dependence is negligible, and the metric is simply that of ordinary
plane polar coordinates in flat space. The one at $r=1/\omega$ is not a coordinate
singularity. The following Maxima code calculates its scalar curvature $R=R\indices{^a_a}$,
which is esentially just the Gaussian curvature, since this is a two-dimensional space.
\begin{listing}{1}
load(ctensor);
dim:2;
ct_coords:[r,theta];
lg:matrix([-1,0],
          [0,-r^2/(1-w^2*r^2)]);
cmetric();   
ricci(true);
scurvature();
\end{listing}
% carousel-singularities.mac
The result is $R=6\omega^2/(1-2\omega^2r^2+\omega^4r^4)$.
This blows up at $r=1/\omega$, which shows that this is not a coordinate
singularity. The fact that $R$ does not blow up at $r=0$ is consistent with our
earlier conclusion that $r=0$ is a coordinate singularity, but would not have been
sufficient to prove that conclusion.\\
(c) The argument is incorrect. The Gaussian curvature is not just proportional to
the angular deficit $\epsilon$, it is proportional to the
limit of $\epsilon/A$, where $A$ is the area of the triangle. The area of the triangle
can be small, so there is no upper bound on the ratio $\epsilon/A$.
Debunking the argument restores consistency with the answer to part b.


%=================================================================================================================================

\vfill\pagebreak


%=================================================================================================================================
% cross-references
\index{gravitational potential|see{potential}}
\index{gravitational red-shift|see{red-shift}}
\index{waves!gravitational|see{gravitational waves}}

%=================================================================================================================================


\noindent\formatlikesection{Photo Credits}\\
\textbf{Cover} Galactic center: NASA, ESA, SSC, CXC, and STScI \qquad % http://hubblesite.org/newscenter/archive/releases/2009/28/image/b/warn/
\cred{hk-in-cabin}{Atomic clock on plane}{Copyright 1971, Associated press, used under U.S. fair use exception to copyright law}
\cred{gravity-probe-a}{Gravity Probe A}{I believe this diagram to be public domain, due to its age and the improbability of its copyright having been renewed}
\cred{hawking-photo}{Stephen Hawking}{unknown NASA photographer, 1999, public-domain product of NASA}%http://en.wikipedia.org/wiki/File:Stephen_Hawking.StarChild.jpg
\cred{eotvos-portrait}{Eotvos}{Unknown source. Since E\"{o}tv\"{o}s died in 1919, the painting itself would be public domain if done from life. Under U.S. law, this makes
        photographic reproductions of the painting public domain}
\cred{inertial-frame}{Earth}{NASA, Apollo 17. Public domain}%http://en.wikipedia.org/wiki/File:The_Earth_seen_from_Apollo_17.jpg
\cred{inertial-frame}{Orion}{Wikipedia user Mouser, GFDL}%http://en.wikipedia.org/wiki/File:Orion_3008_huge.jpg
\cred{inertial-frame}{M100}{European Southern Observatory, CC-BY-SA}%http://en.wikipedia.org/wiki/File:M100.jpg
\cred{inertial-frame}{Supercluster}{Wikipedia user Azcolvin429, CC-BY-SA}% http://en.wikipedia.org/wiki/File:Universe_Reference_Map_%28Location%29_001.jpeg
\cred{artificial-horizon}{Artificial horizon}{NASA, public domain}%http://en.wikipedia.org/wiki/File:VMS_Artificial_Horizon.jpg
\cred{upsidasium}{Upsidasium}{Copyright Jay Ward Productions, used under U.S. fair use exception to copyright law.}
\cred{pound-rebka-photos}{Pound and Rebka photo}{Harvard University. I presume this photo to be in the public domain, since it is unlikely to have had its copyright
        renewed}
\cred{lorentz-portrait}{Lorentz}{Jan Veth (1864-1925), public domain}
\cred{cern-muon-storage-ring}{Muon storage ring at CERN}{(c) 1974 by CERN; used here under the U.S. fair use doctrine}
\cred{tipping-light-cones}{Galaxies}{Hubble Space Telescope. Hubble material is copyright-free and may be freely used as in the public domain without fee, on the condition that NASA and ESA
is credited as the source of the material. The material was created for NASA by STScI under Contract NAS5-26555 and for ESA by the Hubble European Space Agency Information Centre}
\cred{gamma-ray-burst}{Gamma-Ray burst}{NASA/Swift/Mary Pat Hrybyk-Keith and John Jones}
\cred{iijima}{Graph from Iijima paper}{Used here under the U.S. fair use doctrine}
\cred{levi-civita-portrait}{Levi-Civita}{Believed to be public domain. Source: \url{http://www-history.mcs.st-and.ac.uk/PictDisplay/Levi-Civita.html}}
\cred{einsteins-ring}{Einstein's ring}{I have lost the information about the source of the bitmapped image. I would be grateful to anyone who could put me in touch with the
              copyright owners}
\cred{star-magnetic-field-lines}{SU Aurigae's field lines}{P. Petit, GFDL 1.2}%http://en.wikipedia.org/wiki/File:Suaur.jpg
\cred{hole-argument}{Galaxies}{Hubble Space Telescope. Hubble material is copyright-free and may be freely used as in the public domain without fee, on the condition that NASA and ESA
is credited as the source of the material. The material was created for NASA by STScI under Contract NAS5-26555 and for ESA by the Hubble European Space Agency Information Centre}
\cred{chandrasekhar}{Chandrasekhar}{University of Chicago. I believe the use of this photo in this book falls under the fair use exception to copyright in the U.S}%http://www-news.uchicago.edu/releases/98/chandra.jpg
\cred{relativistic-jet}{Relativistic jet}{Biretta et al., NASA/ESA, public domain}% http://en.wikipedia.org/wiki/File:M87_jet.jpg
\cred{dropping-rocks-no-intrinsic-curvature}{Rocks}{Siim Sepp, CC-BY-SA 3.0}% http://commons.wikimedia.org/wiki/File:Diorite.jpg , http://commons.wikimedia.org/wiki/File:Kimberlite.jpg
\cred{comet}{Jupiter and comet}{Hubble Space Telescope, NASA, public domain}
\cred{high-and-low-tides}{Earth}{NASA, Apollo 17. Public domain} % see above
\cred{high-and-low-tides}{Moon}{Luc Viatour, CC-BY-SA 3.0} % http://en.wikipedia.org/wiki/File:Full_Moon_Luc_Viatour.jpg
\cred{heliotrope}{Heliotrope}{ca. 1878, public domain} % http://en.wikipedia.org/wiki/Heliotrope_(instrument)
\cred{triangulation-survey}{Triangulation survey}{Otto Lueger, 1904, public domain}% http://en.wikipedia.org/wiki/File:L-Triangulierung.png
\cred{saddle}{Triangle in a space with negative curvature}{Wikipedia user Kieff, public domain}% http://en.wikipedia.org/wiki/File:Hyperbolic_triangle.svg
\cred{eclipse}{Eclipse}{Eddington's original 1919 photo, public domain}
\cred{spin-torsion-pendulum}{Torsion pendulum}{University of Washington Eot-Wash group, \url{http://www.npl.washington.edu/eotwash/publications/pdf/lowfrontier2.pdf}}
% Emailed them, got reply from Blayne Heckel, relayed back by Charlie Hagedorn, saying "Sounds fine to me."
\cred{asteroids}{Asteroids}{I believe the use of this photo in this book falls under the fair use exception to copyright in the U.S}%http://en.wikipedia.org/wiki/File:Asteroi1.png
\cred{coffee-cup-to-doughnut}{Coffee cup to doughnut}{Wikipedia user Kieff, public domain}% http://en.wikipedia.org/wiki/File:Mug_and_Torus_morph.gif
\cred{coin-with-field-equation}{Coin}{Kurt Wirth, public-domain product of the Swiss government}%http://en.wikipedia.org/wiki/File:Swiss-Commemorative-Coin-1979b-CHF-5-obverse.png
\cred{unruh-photo}{Bill Unruh}{Wikipedia user Childrenofthedragon, public domain} % http://en.wikipedia.org/wiki/File:Wgunruh_phys407.jpg
\cred{accretion-disk}{Accretion disk}{Public-domain product of NASA and ESA}%http://en.wikipedia.org/wiki/File:Accretion_disk.jpg
\cred{killing-portrait}{Wilhelm Killing}{I believe this to be public domain the US, since Killing died in early 1923.}
\cred{killing-vector}{Surface of revolution}{Shaded rendering by Oleg Alexandrov, public domain} % http://en.wikipedia.org/wiki/File:Surface_of_revolution_illustration.png
\cred{cavendish}{Cavendish experiment}{Based on a public-domain drawing by Wikimedia commons user Chris Burks}
\cred{kreuzer-simplified}{Simplified diagram of Kreuzer experiment}{Based on a public-domain drawing by Wikimedia commons user Chris Burks}
\cred{kreuzer}{Kreuzer experiment}{The diagram of the apparatus is redrawn from the paper, and the two graphs are taken directly from the paper. I believe the use of these images in this book falls under the fair use exception to copyright in the U.S.}
\cred{mirror-on-moon}{Apollo 11 mirror}{NASA, public domain}
\cred{static-poynting-vector}{Magnetic fipole}{based on a figure by Wikimedia Commons user Geek3, CC-BY-SA licensed}%http://en.wikipedia.org/wiki/File:VFPt_Dipole_field.svg
\cred{penzias-wilson-antenna}{Penzias-Wilson antenna}{NASA, public domain}% http://grin.hq.nasa.gov/ABSTRACTS/GPN-2003-00013.html
\cred{friedmann-portrait}{Friedmann}{Public domain}% http://en.wikipedia.org/wiki/File:Aleksandr_Fridman.png
\cred{lemaitre}{Lema\^{i}tre}{Ca.~1933, public domain.}% http://commons.wikimedia.org/wiki/File:Lemaitre.jpg
\cred{cmb-geometry}{Cosmic microwave background image}{NASA/WMAP Science Team, public domain}%http://en.wikipedia.org/wiki/File:WMAP_2008.png
\cred{dicke-oblateness}{Dicke's apparatus}{Dicke, 1967. Used under the US fair-use doctrine}
\cred{ligo-and-lisa-sensitivities}{LIGO and LISA sensitivities}{NASA, public domain} % http://en.wikipedia.org/wiki/File:LIGO-LISA.jpg
\cred{pulsar-period-decreasing}{Graph of pulsar's period}{Weisberg and Taylor, \url{http://arxiv.org/abs/astro-ph/0211217}}
% Emailed Weisberg. He complained about lack of attribution. I apologized and made attribution more obvious.
\vfill\pagebreak
\printindex

\emph{Euclidean geometry (page \pageref{euclidean-axioms}):}\\\label{euclidean-summary}
\begin{itemize}
\item[E1] Two points determine a line.
\item[E2] Line segments can be extended.
\item[E3] A unique circle can be constructed given any point as its center and any line segment as its radius.
\item[E4] All right angles are equal to one another.
\item[E5] \emph{Parallel postulate:} Given a line and a point not on the line, exactly one line
          can be drawn through the point and parallel to the given line.\footnote{This is a form known as Playfair's axiom, rather than the version of the
                   postulate originally given by Euclid.}
\end{itemize}

\emph{Ordered geometry (page \pageref{ordered-geometry-axioms}):}\\\label{ordered-summary}
\begin{itemize}
\item[O1] Two events determine a line.
\item[O2] Line segments can be extended: given A and B, there is at least one event such that [ABC] is true.
\item[O3] Lines don't wrap around: if [ABC] is true, then [BCA] is false.
\item[O4] Betweenness: For any three distinct events A, B, and C lying on the same line, we can determine whether or not B is between A and C (and by statement 3, this ordering is unique except for a possible over-all reversal to form [CBA]).
\end{itemize}

\emph{Affine geometry (page \pageref{affine-axioms}):}\\\label{affine-summary}
In addition to O1-O4, postulate the following axioms:
\begin{itemize}
\item[A1] Constructibility of parallelograms: Given any P, Q, and R, there exists S such that [PQRS], and if P, Q, and R are distinct then S is unique.
\item[A2] Symmetric treatment of the sides of a parallelogram: If [PQRS], then [QRSP], [QPSR], and [PRQS].
\item[A3] Lines parallel to the same line are parallel to one another: If [ABCD] and [ABEF], then [CDEF].
\end{itemize}

\emph{Experimentally motivated statements about Lorentzian geometry (page \pageref{lorentz-geometry-postulates}):}\\\label{lorentz-summary}
\begin{itemize}\label{lorentz-geometry-postulates}
\item[L1] \emph{Spacetime is homogeneous and isotropic.} No point has special properties that make it distinguishable from other points, nor is one
                  direction distinguishable from another.
\item[L2] \emph{Inertial frames of reference exist.} These are frames in which particles move at constant velocity if not subject to any forces.
                  We can construct such a frame by using a particular particle, which is not subject to any forces, as a reference point.
\item[L3] \emph{Equivalence of inertial frames:} If a frame is in constant-velocity translational motion relative to an inertial frame, then it is also an inertial frame.
              No experiment can distinguish one inertial frame from another.
\item[L4] \emph{Causality:} Observers in different inertial frames agree on the time-ordering of events.
\item[L5] \emph{No simultaneity:} The experimental evidence in section \ref{sec:time-experiments} shows that
           observers in different inertial frames do not agree on the simultaneity of events.
\end{itemize}

\emph{Statements of the equivalence principle:}\\\label{equivalence-principle-summary}
\begin{itemize}
\item[] Accelerations and gravitational fields are equivalent. There is no experiment that can distinguish one from
the other (page \pageref{equivalence-a-and-g}).
\item[] It is always possible to define a \emph{local} Lorentz
frame in a particular neighborhood of spacetime (page \pageref{equivalence-locally-lorentzian}).
\item[] There is no way to associate a preferred tensor field with spacetime (page \pageref{equivalence-no-preferred-field}).
\end{itemize}

\end{document}
